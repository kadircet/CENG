\documentclass[conference,compsoc]{IEEEtran}
% *** CITATION PACKAGES ***
%
\ifCLASSOPTIONcompsoc
  % IEEE Computer Society needs nocompress option
  % requires cite.sty v4.0 or later (November 2003)
  \usepackage[nocompress]{cite}
\else
  % normal IEEE
  \usepackage{cite}
\fi
\usepackage{graphicx}
\graphicspath{ {../} }
\usepackage{amsmath}
\usepackage{url}
\interdisplaylinepenalty=2500
\usepackage{array}
\usepackage{tikz}
\usetikzlibrary{positioning}
\newcommand{\subparagraph}{}
\usepackage{titlesec}
\setcounter{secnumdepth}{4}
\setcounter{tocdepth}{4}
\setcounter{secnumdepth}{4}
\titleformat{\paragraph}
{\normalfont\normalsize\bfseries}{\theparagraph}{.8em}{}
\titlespacing*{\paragraph}{0pt}{0ex}{0ex}

\begin{document}
\title{Basic Question Answering with Recurrent Models in Turkish}
\author{\IEEEauthorblockN{Kadir Cetinkaya}
\IEEEauthorblockA{Department of Computer Engineering\\
Middle East Technical University\\
Ankara, Turkey\\
Email: e2036457@ceng.metu.edu.tr}}

% make the title area
\maketitle

\begin{abstract}
In this paper we will try and compare different approaches towards basic
Q\&A with recurrent neural networks. We will be using a synthetic dataset
since it is hard to find story, question, answer triplets in Turkish as in
any other language. Our experiments have shown that Selective Memory Networks
are performing a lot better compared to our RNN schemas tried in that project.
Most problematic part of applying RNN methods to these types of NLP problems
is that we are limited in number of steps that we can remember due to vanishing
gradient problem. We were forced to decrease number of words in a sentences
and number of sentences in each story to unrealistic numbers.
\end{abstract}

\section{Introduction}
Q\&A is a problem in which given a story consisting of sentences and a query
one needs to find answer to these query according to the story. Since we take
all three components of the problem as texts it is an NLP problem, and since
text is just a sequence of words or characters, we can use reccurent networks to
model those problems. But due to vanishing gradient problems, we have a limited
ability on the length of the sentences and number of sentences in a story.

In this research we will study several different approaches both using vanilla RNNs
and LSTMs and try to find out how to improve on them what is causing the problems
that prevents our networks from reaching the minima, how to make them converge faster
and generalize better.

\section{Data}
We have generated our data using python, with some predefined object, location and 
actor names, in which actors simply peform some permutation of 4 actions, get, drop,
move and go. Afterwards stories and queries are generated according to those actions
and each query is labeled with its answers according to world state.

Example datum from dataset:\\
hasan parka gitti .\\
sinan eva gitti .\\
hasan servisa gitti .\\
emre tershanea gitti .\\
ali boluma gitti .\\
hasan nerede ?,servis\\
sinan nerede ?,ev\\
emre nerede ?,tershane\\
ali nerede ?,bolum\\

Data is stored in a pickle object, as a list of list of python strings,
each list contains a story and associated questions,answers as shown above.
The whole object contains 1000 stories.

\section{Approach}
\subsection{Basic Approach}
In this approach, we simply encode each sentence of the story into RNNs hidden state.
Afterwards we sum up all of these encodings to generate a mixed hidden state, which
should capture all the information in the story. Finally we feed the query, with the
mixed hidden state as our initial hidden state and aim to get the answer at the last
step of RNN. 

We do not expect these approach to perform any good without a lot of training, since
it threads each sentence equally independent of the question, which makes it harder
to get the relevant information out of it.

\begin{tikzpicture}[shorten >=1pt,->,draw=black!50, node distance=\layersep]
  \tikzstyle{input}=[rectangle,fill=black!25,inner sep=0pt,minimum width=3em,minimum height=3em]
  \tikzstyle{RNN}=[rectangle,fill=red!25,inner sep=0pt,minimum width=1.5em,minimum height=8em]
  \tikzstyle{output}=[rectangle,fill=black!25,inner sep=0pt,minimum width=3em,minimum height=3em]
  \node[input,label=above:{NxT}] (h0) {Story};
  \node[RNN,right=2em of h0] (h1) {};
  \node[RNN,right=2em of h1,label=above:{NxH}] (h2) {Encoding};
  \node[output,right=2em of h2,label=above:{1xH}] (h3) {Sum of Encoding};
  \node[input,below=5em of h3,label=below:{1xT}] (h4) {Query};
  \node[RNN,left=2em of h4,label=below:{1xH}] (h5) {};
  \node[output,left=3em of h5,label=below:{1xV}] (h6) {Output};
  \path 
    (h0) edge node{$W_e$} (h1)
    (h1) edge[loop above] node{$W_{h_1}$} (h1)
    (h1) edge node{$W_{h_1}$}(h2)
    (h2) edge node{$\Sigma$} (h3)
    (h3) edge node{$W_{h_2}$} (h5)
    (h4) edge node{$W_e$} (h5)
    (h5) edge node{$W_{vocab}$} (h6);
\end{tikzpicture}

\subsection{Concatenation of the Encodings}
Instead of adding encodings elementwise we will concatenate them including the
query encoding and try to train a fully connected layer, which will try to 
figure out the answer to query using all information present in the story.
It should perform elicitation of irrelevant information implicitly. That is
what we expect to from the fully connected layer to learn.

\begin{tikzpicture}[shorten >=1pt,->,draw=black!50, node distance=\layersep]
  \tikzstyle{input}=[rectangle,fill=black!25,inner sep=0pt,minimum width=3em,minimum height=3em]
  \tikzstyle{RNN}=[rectangle,fill=red!25,inner sep=0pt,minimum width=1.5em,minimum height=8em]
  \tikzstyle{output}=[rectangle,fill=black!25,inner sep=0pt,minimum width=3em,minimum height=3em]
  \node[input,label=above:{NxT}] (h0) {Story};
  \node[RNN,right=2em of h0] (h1) {};
  \node[RNN,right=2em of h1,label=above:{NxH}] (h2) {Encoding};
  \node[input,below=7em of h0,label=below:{1xT}] (h4) {Query};
  \node[RNN,right=2em of h4,label=below:{1xH}] (h5) {};
  \node[output,right=3em of h5,label=below:{1x((N+1)*H)}] (h3) {Concatted Story and Query};
  \node[output,right=3em of h2,label=below:{1xV}] (h6) {Output};
  \path 
    (h0) edge node{$W_e$} (h1)
    (h1) edge[loop above] node{$W_{h_1}$} (h1)
    (h1) edge node{$W_{h_1}$}(h2)
    (h2) edge node{Concat} (h3)
    (h5) edge[loop below] node{$W_{h_2}$} (h5)
    (h5) edge node{Concat} (h3)
    (h4) edge node{$W_e$} (h5)
    (h3) edge node{$W_{vocab}$} (h6);
\end{tikzpicture}

\subsection{Selective Memory}
In our basic approach we added encodings of story sentences uniformly, now instead
we will also encode our query beforehand and add story encodings in accordance with
similarity to that encoded query, which should perform better by selecting similar
story sentences and electing irrelevant ones.

The selecting will be done by taking dot product of each story sentence's encoding
with query sentence's encoding, which will give their similarity score. Then we 
will perform softmax on those scores to convert them into probabilities. Afterwards
we will sum all encodings for the story using those weigths.

The idea is quite similar to memory networks, more can be done by inserting additional
hops by performing the above mentioned steps iteratively and making the prediction at
the end, which should enable network to learn even higher dependincies to to story
sentences like where was an object before it was there, or who put the object to its
current location etc.


\begin{tikzpicture}[shorten >=1pt,->,draw=black!50, node distance=\layersep]
  \tikzstyle{input}=[rectangle,fill=black!25,inner sep=0pt,minimum width=3em,minimum height=3em]
  \tikzstyle{RNN}=[rectangle,fill=red!25,inner sep=0pt,minimum width=1.5em,minimum height=8em]
  \tikzstyle{output}=[rectangle,fill=black!25,inner sep=0pt,minimum width=3em,minimum height=3em]
  \node[input,label=above:{NxT}] (h0) {Story};
  \node[RNN,right=2em of h0] (h1) {};
  \node[RNN,right=2em of h1,label=above:{NxH}] (h2) {Encoding};
  \node[input,below=7em of h0,label=below:{1xT}] (h4) {Query};
  \node[RNN,right=2em of h4,label=below:{1xH}] (h5) {};
  \node[output,right=7em of h5,label=right:{1xH}] (h3) {Focused Encoding};
  \node[RNN,below=3em of h3,label=below:{1xH}] (h6) {};
  \node[output,below=3em of h6,label=below:{1xV}] (h7) {Output};
  \path 
    (h0) edge node{$W_e$} (h1)
    (h1) edge[loop above] node{$W_{h_1}$} (h1)
    (h1) edge node{$W_{h_1}$}(h2)
    (h2) edge node{Dot Product} (h3)
    (h5) edge[loop below] node{$W_{h_2}$} (h5)
    (h6) edge[loop right] node{$W_{h_2}$} (h6)
    (h5) edge node{Dot Product} (h3)
    (h3) edge node{$W_{h_2}$} (h6)
    (h4) edge node{$W_e$} (h5)
    (h4) edge node{$W_e$} (h6)
    (h6) edge node{$W_{vocab}$} (h7);
\end{tikzpicture}

\section{Experiments}
\subsection{Dataset Convertion}
Our dataset comes as a primitive python object, we first need to convert it to
a numpy array/matrix. To do so, we read all the dataset and convert it into a
NxK matrix, where N is the total number of queries and K is the predetermined
maximum number of words in each story + query + answer, which is 37. We simply
read a story and all of its queries, then take each query as a seperate datum
prepend the story to it and move forward. We have 4275 queries in total, the
last 1000 of them will be used as test data, so we will be training on a
3275x37 data.

\subsection{Word2Vec}
Also instead of using strings, we use word2vec notation there are 159 different
words in total in our dataset, so each word will be represented by an integer in
range [0,158]. Then we perform embedding of all these words into a K dimensional
space, so that instead of just representing them with integers and highly 
discriminating between them, we will be able to learn similarities between those
words and perform operations based more on those meanings, features of the word not
according to just some integer.

\subsection{Network}
For all those approaches, we tried to keep the underlaying architecture same, so
we have a RNN with 512 hidden units, we set also the cell type to perform either
LSTM or vanilla RNN operations during the loss calculations. We use 2 different
parameter set for story sentences and query sentence, but word2vec parameters
are shared between them.

Due to differences in approaches, the maximum unfoldings and how the loss and
gradients get calculated changes from approach to approach, approach specific
implementations are detailed in their own chapter.

\subsection{Basic Approach}
\subsubsection{Implementation}
In that experiment we will calculate encodings for each sentence of the story
then combine them by simply adding and feed them for the encoding of the query.
Then decode the answer from the encoding of the query. So we will use 7 as our
unrolling factor, which is the maximum length of a sentence, including start
and end symbols. For sentences with less words we will perform padding with a
special null character. For the query sentence we perform unrolling with a factor
of 5, since there are less words in questions.

\subsubsection{Results}
We tried to overfit a sample set of 50 stories on that network, within 200 iterations
we get a loss of 0.1304 which is more than enough for overfitting of the network, since
we can get a training accuraccy of 1.0 after the loss goes lower than 0.5 for that small
dataset.

\includegraphics[scale=.3]{basic_loss}

For the training phase we performed 4000 iterations, which yielded a loss around 0.37
within those iterations we gathered 0.87 training accuracy and 0.38 test accuracy.

So even if we get such a good training accuracy with so little training, it seems like
our model does not generalize. There are a few reasons behind that which are obvious,
we treat each sentence of the story equally important, we just sum the encoded story
and feed it directly to our query, therefore network also tries to learn in which
story it should focus, which is quite hard with that little dataset and iterations.
Therefore it fails to generalize and instead tries to memorize the training data,
which gets it to a better minima.

\subsection{Concatenation of Encodings}
\subsubsection{Implementation}
In that experiment we will calculate encodings for each sentence of the story and
encoding of the query sentence, then combine them by simply concatenating each
encoding into one big vector and try to learn a fully conencted layer which tries
to extract the answer using that vector. It will perform selection of story sentences
implicityl during that step by learning to focus on different parts of the concatenated 
vector with respect to query's encoding. So as output we simply generate a 1xV vector
which is the probability of answer being $i^{th}$ word.

So we will use 7 as our unrolling factor, for the embedding of stories which is the 
maximum length of a sentencein story, including start and end symbols. 
For sentences with less words we will perform padding with a special null character. 
For the query sentence we perform unrolling with a factor of 5, since there are less 
words in questions.

\subsubsection{Results}
We tried to overfit a sample set of 50 stories on that network, within 100 iterations
we almost overfitted the data completely with a loss of 0.0372, it overfitted faster than
our Basic Approach. Reached baseline's performance within 60 iterations. So we converged 3
times faster. Which can be explained by networks ability to get all encodings of the story
sentences directly and does not need to learn about how to extract which part of the encoding
corresponds to which sentence.

\includegraphics[scale=.3]{concat_loss}

For the training phase we performed 4000 iterations, which yielded a loss around 1.5
which is quite far from our performance compared to basic approach. 
Within those iterations we gathered 0.66 training accuracy and 0.31 test accuracy.
Which shows there is still room for improvement, since in the last layer of our network
we have a fully connected layer with huge dimensions (11*HIDDENSIZE), our network was not
able to converge to minima with these number of iterations. But despite the lack of that
convergence we get a very close score to our basic approach. Which implies that our new
network can generalize better and does not suffer from overfitting as much as our basic
approach.

That network can give a lot better results by tuning the hyperparameters and performing
more iterations during the training phase. As can be seen from the loss graph after 2000
iterations network was not able to go lower in the loss layer, which is most likely caused
by a big learning rate, that resulted in oscilations around the minima.

\subsection{Selective Memory}
\subsubsection{Implementation}
In that experiment we will calculate encodings for each sentence of the story
and query. Then we will take dot product of each story sentences representation
and query's encoding to get a similarity scores, afterwards we perform softmax on those
scores to convert them into probability weigths to be used for determining the focus
of the network on each memory. Similar to previous approaches we will perform an
unrolling of 7 units for story and 5 units for query sentence. Afterwards that
hidden representation will be fed to our RNN as initial hidden state with query
as input. Then we will get the output from our fully connected layer.

This approach is quite similar to most of the SOTA approaches, we expect it to perform
a lot better than previous ones. It should be converging faster than concatenation
method since we have a lot less parameters, also should be able to generalize better
than our basic approach since instead of summing up representations uniformly we focus
on sentences that are more likely to be relevant to our query.

\subsubsection{Results}
We tried to overfit a sample set of 50 stories on that network, within 400 iterations
we almost overfitted the data completely with a loss of 0.03952, it overfitted slower than
our Basic Approach. Reached baseline's performance within 300 iterations, but in last 100
iteration it performed far more faster than our Basic Approach and converged more than 5
times. It is most likely to be caused by learning of the query encoding parameters. Since
we use a different encoding set for query and it is importance is quite high in correspondence
to our Basic Approach, it either keeps or discards sentences of the story depending on that
encoding. After it has been learned, the network converges a lot faster.

\includegraphics[scale=.3]{selective_loss}

For the training phase we performed 6000 iterations, which yielded a loss around 0.05
which is far more superior to our basic approach, 7 times better to be precise. 
Within those iterations we gathered 0.995 training accuracy and 0.52 test accuracy.
Which shows overfitting, since we almost answered every query in the training set
but get only half of the questions correct in the test case. Which was not possible
in our Basic Approach, it was only able to get 0.87 training set accuracy and could
not overfit the dataset, despite having same number of parameters. Which shows
our Selective Memory Network approach can perform a lot better by introducing more
stochastic methodolgies to overcome overfitting.

\pagebreak
\section{Conclusion}
As can be seen from the experiments, Selective Memory Networks are best choice and
by introducing more stochastic methods to the network one can make sure the overfitting
is less severe and generalization can be achieved. In comparison to mentioned concatenation
method, we are likely to converge faster but suffer from overfitting. So the concatenation
networks can perform quite close to selective memory networks given more resources to
perform a lot more training. Our basic approach is most likely to fail in any real world
data since it was only able to overfit to that spesific dataset and didn't generalize
to even test data which come from very same nature as the training. To summarize the
study we provide a table showing our accuracy for test and training data:

\begin{tabular}{|l|l|l|}
  \hline
  Approach & Training Acc & Test Acc \\ \hline
  Basic Approach & 0.87 & 0.38 \\ \hline
  Concatenation of Encodings & 0.66 & 0.31 \\ \hline
  Selective Memory & 0.995 & 0.52 \\ \hline
\end{tabular}

\section{Future Work}
\subsection{Selective Memory}
One should try adding dropout and tuning of hyperparameters to overcome the overfitting
problem of the architecture to see if it is able to generalize better.

\subsection{Concatenation of Encodings}
The hyperparameters of the network should be tuned to allow it perform more learning,
in our experiments we always get stuck after 2000 iterations, but it should be possible
to learn better, overfit at least since we have only 3275 datapoints and 11*128*256 parameters
at the fuly connected layer, which should be more than enough to overfit.

\section{Project Materials}
All of the material used in this project is available under \url{https://github.com/kadircet/CENG/tree/master/783/project}.
It includes the dataset generator, source codes for the network, both architecture and utilities,
they are all implemented in python without any existing framework, during the homeworks
of the course METU-CENG783.

\pagebreak
\begin{thebibliography}{1}
\bibitem{TowAIComp}
Jason Weston, Antoine Bordes, Sumit Chopra, Alexander M. Rush, Bart van Merriënboer, Armand Joulin, Tomas Mikolov,
    \emph{Towards AI-Complete Question Answering: A Set of Prerequisite Toy Tasks}
\bibitem{MemNetw}
Jason Weston, Sumit Chopra, Antoine Bordes, 
    \emph{Memory Networks}
\bibitem{MSteve}
    Merity, Steve. \emph{Question Answering on the Facebook BAbi Dataset Using Recurrent Neural Networks}
\bibitem{Ceng783}
    Kalkan, Sinan. \emph{METU CENG 783 Lecture Notes}
\end{thebibliography}
\end{document}

