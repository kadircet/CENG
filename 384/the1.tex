\documentclass[12pt]{article}
\usepackage[utf8]{inputenc}
\usepackage{float}
\usepackage{amsmath}


\usepackage[hmargin=3cm,vmargin=6.0cm]{geometry}
%\topmargin=0cm
\topmargin=-2cm
\addtolength{\textheight}{6.5cm}
\addtolength{\textwidth}{2.0cm}
%\setlength{\leftmargin}{-5cm}
\setlength{\oddsidemargin}{0.0cm}
\setlength{\evensidemargin}{0.0cm}

%misc libraries goes here
\usepackage{tikz}
\usetikzlibrary{automata,positioning}

\begin{document}

\section*{Student Information } 
%Write your full name and id number between the colon and newline
%Put one empty space character after colon and before newline
Full Name :  Kadir CETINKAYA \\
Id Number :  2036457

% Write your answers below the section tags
\section*{Answer 1}
\begin{itemize}
	\item \textbf{Sine Wave}\\
	The sine wave gives a smooth and continous sound, with the same frequency all along.
	Because it consists of only one sinusoid.

	\item \textbf{Square Wave}\\
	Since Square Wave is an odd function, it is generated by adding the odd multiples of
	220Hz, therefore it has different frequencies as well and louder due to different amplitudes
	coming from all those frequencies.

\item \textbf{Sawtooth Wave}\\
Since Sawtooth Wave contains all multiples of the frequencies in its fourier expansion,
the sound contains more frequencies than square wave, but it is not as loudy as square
wave, which is quite expected since we add more sinusoids we expect their amplitude to be
lesser than before.

\item \textbf{Triangle Wave}\\
It is almost same as the square wave only difference is it is not as loudy as square wave,
which implies the amplitudes of the frequencies are lower. The similarity is caused by oddity of
triangle function, it also contains odd multiples of 220Hz in it.
\end{itemize}


\section*{Answer 2}
Since it takes the abs of the sinusoid, it changes the signal from sinusoid with 220Hz to
a signal with 440Hz. Therefore it has twice the frequency, so sound is different.

\section*{Answer 3}
As n increases $g_n(x)$ starts to include almost every frequency. From fourier transform of impulse
function we know that it is constant 1 in frequency domain, therefore it includes every frequency. 
Therefore, we approximate the impulse function with frequency $f$ as $n$ increases. 
This $g_n(x)$ can be used to sample a signal, with a sampling rate of $f$ just by multiplying the
signal to be sampled with $g_n(x)$.

\section*{Answer 4}
Function is periodic with $T=2$, therefore $2L=2, L=1$. Also $f(t)=0$ for $1\leq t\leq 2$, so
when calculating we can just look at the interval $\lbrack 0,1\rbrack$ instead of $\lbrack 0,2\rbrack$.
We can calculate $$c_0=\dfrac{1}{2}\int_{0}^{1}f(t)dt=\dfrac{1}{2}$$ and
$$c_k = \dfrac{1}{2}\int_{0}^{1}f(t)e^{ik\Pi t}dt=\dfrac{-i}{2k\Pi}(e^{ik\Pi}-1)$$ as $n$ increases
we aprroximate $f(t)$ more, the oscilations around $1$ and $0$ gets smaller and smaller, after $2000\leq n$
the oscillations become so small, they can not be seen by human eye. Also with low values of $n$ there are spikes
around transitions from $1$ to $0$ and $0$ to $1$ which are caused by usage of less complex exponentials, which
is not enough to approximate the $f(t)$.
\end{document}

