\documentclass[12pt]{article}
\usepackage[utf8]{inputenc}
\usepackage{float}
\usepackage{amsmath}


\usepackage[hmargin=3cm,vmargin=6.0cm]{geometry}
%\topmargin=0cm
\topmargin=-2cm
\addtolength{\textheight}{6.5cm}
\addtolength{\textwidth}{2.0cm}
%\setlength{\leftmargin}{-5cm}
\setlength{\oddsidemargin}{0.0cm}
\setlength{\evensidemargin}{0.0cm}

%misc libraries goes here
\usepackage{tikz}
\usetikzlibrary{automata,positioning}

\begin{document}

\section*{Student Information } 
%Write your full name and id number between the colon and newline
%Put one empty space character after colon and before newline
Full Name : Kadir Cetinkaya \\
Id Number : 2036457 \\

% Write your answers below the section tags
\section*{Answer 1}

\subsection*{a.}
The words in that language has the form $x^ny^nz^k$ or $x^ny^kz^n$ where $n\ge 0, k\ge 0$.

\subsection*{b.}
The words in the langauge has the form $a\$ba^Rc$, where $a,b,c\in \{x,y\}^*$ and 
$a^R$ is the reverse of $a$.

\subsection*{c.}
$(x(xxxxxxxxx)^*)+((z+y)(z+y))^*$



\section*{Answer 2}

\subsection*{a.}
Let's assume the language is CFL, then according to pumping lemma $\exists n>0$ such that,
$\forall z\in L_1$ such that $|z|\ge n$, $\exists u,v,w,x,y\in \{1,2\}^*$ such that $z=uvwxy$,
$|vwx|\le n$, $|vx|>0$, $uv^iwx^iy\in L_1$, $\forall i\ge 0$.

Let's examine $z=1^n2^n2^n1^n2^{2n}\in L_1$:\\

i) Let's say string $vx$ contains any $1$'s. Since $|vwx|\le n$, those $1$'s must be only in the
first block or the second block of $1$'s, therefore when we pump them up we can only increase
number of $1$'s in the one block, which would make our word out of the language. So we get a
contradiction, therefore $vx$ contains no $1$'s.

ii) Let's say string $vx$ contains any $2$'s, by part i) we know that it does not contain any
$1$'s, and since $|vwx|\le n$ those $2$'s can only be in the middle part of $z$ or at the end of
$z$. Let's say those $2$'s are in the middle, then if we pump them up even without violating
the reverse string property, we can't increase the $2$'s at the end which would make the new
word out of the language, so $2$'s in the $vx$ must be coming from the end of the $z$. In
that case if we pump them up, since we can't increase the length of the first part of the
string we would again make the new word out of the language. So we get a contradiction again.
Therefore $vx$ contains no $2$'s.

By part i) and ii) since $vx$ can't contain any $1$'s or $2$'s and those are the only letters
from our alphabet, $|vx|=0$. But that's against the pumping lemma so our assumption is wrong.
Therefore the language $L_1$ is not a CFL.

\subsection*{b.}
$A\cap B=\{e\}$, therefore $A\setminus B=A\setminus \{e\}$. So we can generate $L_2$ with the
following grammer:\\
$$S\rightarrow xAz | yBz$$
$$A\rightarrow xAz | B$$
$$B\rightarrow yBz | e$$

where $e$ inticates empty string.

\subsection*{c.}
Let's assume the language is CFL, then according to pumping lemma $\exists n>0$ such that,
$\forall z\in L_3$ such that $|z|\ge n$, $\exists u,v,w,x,y\in \{a,b,c\}^*$ such that $z=uvwxy$,
$|vwx|\le n$, $|vx|>0$, $uv^iwx^iy\in L_3$, $\forall i\ge 0$.

Let's examine $z=a^nb^nc^{n*n}\in L_3$:\\

i) Let's say string $vx$ contains any $a$'s, since $|vwx|\le n$ then $vx$ can't contain any $c$.
So, if we pump up $v$ and $x$ the number of $a$'s would increase, but number of $c$'s wouldn't
change which would make the new word out of the language, so $vx$ can't contain any $a$'s.

ii) Let's say string $vx$ contains at least 1 $b$, since $|vwx|\le n$ it can't contain more than 
$n-1$ $c$'s. So if we pump up by one $v$ and $x$, the number of $b$'s increase by at least one,
therefore number of $c$'s must increase at least by $n$, but that's not possible since we have
at most $n-1$ $c$'s, therefore $vx$ can't contain any $b$'s.

iii) By part i) and ii) $vx$ can contain only $c$'s, let's say $vx$ has at least 1 $c$, since
it doesn't contain any $a$ or $b$ if we pump down $v$ and $x$ the new word would have less $c$'s
but same amount of $a$'s and $b$'s which would make the new word out of the language. So $vx$
doesn't contain any $c$'s.

By part i), ii) and iii) $vx$ can't contain any $a$, $b$ or $c$ and those are the only letters
from our alphabet, therefore $|vx|=0$, which contradicts with pumping lemma. Therefore our
assumption is wrong. So the language $L_3$ is not a CFL.

\subsection*{d.}
No word in $L$ starts with $2$, so we have every word starting with $2$ in $L_4$, for words
starting with $1$ we have 3 cases, if the string contains $21$ as a substring than it can't be
in $L$, therefore it is in $L_4$. If it doesn't than it means word is of the form $1^n2^m$ and
for it to be in $L_4$ we must have either strictly more $1$'s than $2$'s or strictly more $2$'s
than twice the amount of $1$'s. Also since the empty string is in $L$, we can't have empty string
in $L_4$. We can express it like this:

$$S\rightarrow 2G | 1S_1 | S_22 | 1G21G$$
$$G\rightarrow 1G | 2G | e$$
$$S_1\rightarrow 1S_12 | 1S_1 | e$$
$$S_2\rightarrow 1S_222 | S_22 | e$$

where $e$ indicates empty string.


\section*{Answer 3}


\subsection*{a.}
Let $M=(Q, \Sigma, \Gamma, \delta, q_0, F)$, where $Q=\{q_0,q_1,q_2,q_3\}$, $\Sigma =\{0,1\}$,
$\Gamma =\{A,B\}$, $F=\{q_3\}$ and $\delta =
\begin{aligned}[t]
	& \{(q_0, 0, \epsilon, q_1, A), (q_0, x, \epsilon, q_3, \epsilon), (q_0, 1, \epsilon, q_1, B),
	(q_1, x, y, q_1, yy), (q_1, 0, A, q_2, A), \\ 
	& (q_1, 1, B, q_2, B), (q_2, x, y, q_2, \epsilon),
	(q_2, 0, A, q_3, \epsilon), (q_2, 1, B, q_3, \epsilon)\ | x\in \{0,1\}, y\in \{A,B\}\}.
\end{aligned} $
\\
Basically, it starts in state $q_0$ with empty stack, reads the first character of the input
and either nondeterministically jumps to $q_3$ with empty stack, for matching the strings with
length 1, or pushes $A$ to stack if it's $0$, pushes $B$ otherwise and goes on to state $q_1$. 
In that state it processes the input upto the middle point and pushes a copy of the top element 
to the stack, when it reads a character same as the first character of the input, it controls
that by the top element of the stack since we always push $A$ if the first element is $0$
and $B$ otherwise, it jumps to $q_2$ nondeterministically without pushing a new element.
Also in $q_1$ since it pushes a new symbol to the stack for every character except the middle
one, we count the number of characters in the first half. In state $q_2$, we pop one element
from the stack for every character of the input and jump to $q_3$ nondeterministically with
popping the top element if it's equal to first character of the input, checked in the same 
manner as above. The state $q_3$ is final and there is no outgoing transition, therefore
the whole string must be processed and the stack must be empty for the word to be accepted
and it is only possible if input has length 1 or transition to $q_2$ occured exactly at 
the middle character and transition to $q_3$ occured exactly at the last character of the 
input, which can happen only on the characters same as the first character of the input.

\subsection*{b.}
$(q_0, 01000, \epsilon )\vdash_M (q_1, 1000, A)\vdash_M (q_1, 000, AA)\vdash_M (q_2, 00, AA)\vdash_M
(q_2, 0, A)\vdash_M (q_3, \epsilon, \epsilon)$

\subsection*{c.}
$(q_0, 00100, \epsilon )$ in start we have 2 possibilities, we can either go to $q_1$ or $q_3$,
it is obvious that if we jump to $q_3$ we will get stuck, so let us check the case we go for
$q_1$;

$(q_0, 00100, \epsilon )\vdash_M (q_1, 0100, A)$, in that point we have 2 possible paths either to $q_2$
or $q_1$, lets examine the $q_2$ path first.

$(q_0, 00100, \epsilon )\vdash_M (q_1, 0100, A)\vdash_M (q_2, 100, A)\vdash_M (q_2, 00, \epsilon)$
we get stuck since there is no transition from $q_2$ with empty stack, so let us look at the
other option.

$(q_0, 00100, \epsilon )\vdash_M (q_1, 0100, A)\vdash_M (q_1, 100, AA)\vdash_M (q_1, 00, AAA)$, in that
point we also have 2 possible paths but since there are more elements in stack than the remaining
characters of the input, and we pop at most one element per each input character no matter
which path we take, we won't be able to clear the stack, so we will get stuck or finish in a
nonaccepting state.

Since there is no computation acceptiong $00100$, it is not in the $L(M)$.

\section*{Answer 4}

\subsection*{a.}
\begin{itemize}
	\item 
		$L_1\cup (L_2\setminus R) =L_1\cup (L_2\cap \overline{R})$, since set difference is 
		intersection with complement of the set. $L_2\cap \overline{R}$ is a CFL since
		complement of a regular language is regular and intersection of a CFL and a regular
		language is also a CFL. $L_1\cup (L_2\cap \overline{R})$ is also a CFL since union
		of two CFL's is again a CFL and since it is equal to $L_1\cup (L_2\setminus R)$,
		given language is also a CFL.
	\item
		Let $R,L_1$ and $L_2$ be defined on the alphabet $\Sigma$, moreover let $R=\Sigma^*$.
		Then, $R\setminus (L_1\cup L_2) = R\cap \overline{(L_1\cup L_2}) = \overline{(L_1\cup L_2)}$.
		Since $R$ is the set of all words that can be generated in that alphabet and by the
		definition of set difference. $L_1\cup L_2$ is a CFL since CFL's are closed under union,
		but their complement is not necessarilly a CFL. Therefore given language is not
		necessarilly a CFL.
\end{itemize}

\subsection*{b.}
Yes, $G$ is ambiguous. Because we can generate word $xxy$ both using the ruleset,\\
$S\rightarrow XY \rightarrow XxY \rightarrow xxY \rightarrow xxy$ and \\
$S\rightarrow xxY \rightarrow xxy$.

\begin{tikzpicture}[shorten >=1pt, on grid, auto]
	\node (S0) {S};
	\node (X1) [below left=of S0] {X};
	\node (Y1) [below right=of S0] {Y};
	\node (X2) [below left=of X1] {X};
	\node (x2) [below right=of X1] {x};
	\node (y2) [below =of Y1] {y};
	\node (x3) [below =of X2] {x};
	\path[->]
	(S0) edge (X1)
			edge (Y1)
	(X1) edge (X2)
	(X1) edge (x2)
	(Y1) edge (y2)
	(X2) edge (x3);
\end{tikzpicture}
\hspace{3cm}
\begin{tikzpicture}[shorten >=1pt, on grid, auto]
	\node (S0) {S};
	\node (X1) [below left=of S0] {xx};
	\node (Y1) [below right=of S0] {Y};
	\node (y2) [below =of Y1] {y};
	\path[->]
	(S0) edge (X1)
			edge (Y1)
	(Y1) edge (y2);
\end{tikzpicture}

\subsection*{c.}
Let $L=\{aa^R2^{|a|}|a\in \{1,2\}^*\}$, M can recognize that language by pushing the input
while reading $a$ to one of its stacks, than nondeterministically jumping to reading $a^R$
part of the input and while reading that part, it can pop from the stack it was pushing
before and compare the input character with the top of the stack, while pushing into
its other stack an element to measure the length of the word $a$ and than it can
nondeterministically jump to reading $2^{|a|}$ part of the input by popping from its
latter stack for every $2$ it had read from the input. Whereas N, a PDA with one stack
can't recognize that string as in question 2 part a, that language is shown to be not context
free. Since we can recognize a language with M that we weren't able to recognize with N,
M is more powerful than N.

\end{document}

​

